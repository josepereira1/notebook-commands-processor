\documentclass[4apaper]{report}
\usepackage{graphicx}
\usepackage[utf8]{inputenc}
\usepackage[portuguese]{babel}
\usepackage{a4wide}
\title{Projeto de Sistemas Operativos}
\author{Henrique José Carvalho Faria a82200\and José André Martins Pereira a82880 \and Ricardo André Petronilho a81744}
\date{\today}

\begin{document}

\maketitle

\begin{abstract}
No âmbito da Unidade Curricular de Sistemas Operativos (SO), do Mestrado integrado em Engenharia Informática da Universidade do Minho, foi nos proposto a elaboração de um projeto que consiste em construir um sistema de processamento de notebooks, que mistura fragmentos de código, resultados da execução, e documentação. Neste contexto, um notebook é um ficheiro de texto que depois de processado é modificado de modo a incorporar resultados da execução de código ou comandos nele contidos. Este trabalho foi realizado na linguagem C.
\end{abstract}

\tableofcontents

\newpage

\chapter{Introdução}
\label{sec:introducao}

O trabalho foi dívidido em duas fases:

\begin{itemize}
	\item Na primeira fase tratou-se da leitura do ficheiro notebook, e intrepertação dos dados.

	\item A segunda fase passou por executar os respetivos comandos, podendo estes serem encadeados e por fim imprimir os seus outputs no ficheiro notebook.
\end{itemize}

\newpage

\section{Funcionamento do Processador de notebooks}

Inicialmente faz-se um check-out de toda a informação contida no ficheiro para um buffeer. Após o preenchimento desse buffer, faz-se o parser para a nossa estrutura, a razão desta estrutura foi para facilitar o tratamento dos dados, sendo esta composta por:
\begin{verbatim}
/**\brief estrutura que armazena um comando */
typedef struct TCD_COMANDO{

	char* comentario;
	char* nome;
	char** argumentos;
	int encadeado;

} TCD_COMANDO;

/**\brief estrutura que armazena todos os comandos */
typedef struct TCD_COMANDOS{

	int dim;
	int ocupados;
	TCD_COMANDO* comandos;

} TCD_COMANDOS;

\end{verbatim}

Onde o parâmetro \texttt{comentario} é o comentário referente a este comando, \texttt{nome} é nome do comando, \texttt{argumentos} são os argumentos do comando, \texttt{encadeado} verifica se o comando é encadeado com o anterior, \texttt{dim} é a dimensão da estrutura, \texttt{ocupados} é o número de comandos, \texttt{comandos} é o ponteiro que aponta para a estrutura \texttt{TCD\_COMANDO}. 

Depois de termos o parser feito, percorremos a estrutura do comando na posição 0 até ao fim e para cada posição, invocamos a função execEncadeada, esta função distingue um comando não encadeado, de um encadeado consecutivo e de um encadeado não consecutivos guardando num array auxiliar por ordem inversa, encadeando dos comandos independentes do número de comandos ou de este serem consecutivos ou não. De seguida procedemos à criação de um processo filho para executar cada um dos comandos encadeados e guardar o output num pipe anónimo. 


\subsection{Dificuldades:}

Uma das dificuldades do trabalho foi a elaboração do parser do ficheiro notebook, visto que existem problemas de memória e raramente existem erros com a alocação. Tenta-mos resolver este problema de diversas formas, mas sem sucesso, devido à aproximação da data de entrega, não conseguimos colocar um parser 100\% funcional.

\subsection{Estratégias}

A estratégia inicial para facilitar a manipulação dos dados lidos pelo ficheiro foi a criação de uma estrutura que guarda os diferentes parâmetros de cada comando: comentário, nome do comando, argumentos e flag que indica se o comando é encadeado com outros. Isto vai facilitar, pois vai ser mais intuitivo para a leitura do código. Esta estrutura foi implementada como um tipo abstrato de dados, para que haja maior modularidade.

\newpage

\chapter{Conclusão}

A realização deste projeto, propôs um grande desafio para o nosso grupo de trabalho tornando-se muito enriquecedor, pois desenvolvemos o raciocínio lógico e os métodos de trabalho.

Terminado este trabalho, alcançou-se o objectivo inicial do projeto. 

No decorrer do mesmo surgiram algumas dificuldades, como haveriamos de encadear os comandos, dificuldades essas que se foram dissipando, com aplicação dos conteúdos programáticos abordados nas aulas teóricas e o apoio indespensável dos docentes, como forma a ultrapassar e resolver todos os problemas surgidos.

Sugerimos assim, a continuidade da realização destes projetos, como forma a desensolver espírito criativo e funcional indespensável à prática da nossa Profissão. 

\end{document}